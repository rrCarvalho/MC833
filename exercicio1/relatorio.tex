\documentclass[a4paper,10pt,oneside,final,titlepage,onecolumn]{article}

\usepackage{ucs}
\usepackage[portuguese]{babel}
\usepackage[utf8x]{inputenc}
\usepackage[T1]{fontenc}
\usepackage{textcomp}

\usepackage{listings}
\usepackage{color}

\definecolor{dkgreen}{rgb}{0,0.6,0}
\definecolor{gray}{rgb}{0.5,0.5,0.5}
\definecolor{mauve}{rgb}{0.58,0,0.82}

\lstset{frame=tb,
  language=bash,
  aboveskip=3mm,
  belowskip=3mm,
  showstringspaces=false,
  columns=flexible,
  basicstyle={\scriptsize\ttfamily},
  numbers=none,  
  breaklines=true,
  breakatwhitespace=true
  tabsize=3
}

\title{Exercício 1 de MC833 --- Programação em Redes de Computadores}
\author{Raul Rabelo Carvalho, 105607, turma A}

\begin{document}

\maketitle

\section{ping}

\subsection{}
\paragraph{}A opção \mbox{-c} do ping serve para escolher o número de vezes que o programa irá enviar um \emph{echo request} para a máquina-alvo. Caso esta opção não seja passada, o ping continuará a enviar pedidos de resposta até que o usuário envie um sinal de parada (\emph{Ctrl-C}) ao programa.
\paragraph{}O tempo de ida e volta mínimo para um pedido de resposta à uma máquina no endereço \mbox{www.cam.ac.uk} foi 255 ms; o tempo máximo, 256 ms; e o tempo medio, 255,3 ms.
\begin{lstlisting}
PING www.cam.ac.uk (131.111.150.25) 56(84) bytes of data.
64 bytes from primary.admin.cam.ac.uk (131.111.150.25): icmp_seq=1 ttl=47 time=256 ms
64 bytes from primary.admin.cam.ac.uk (131.111.150.25): icmp_seq=2 ttl=47 time=255 ms
64 bytes from primary.admin.cam.ac.uk (131.111.150.25): icmp_seq=3 ttl=47 time=255 ms
64 bytes from primary.admin.cam.ac.uk (131.111.150.25): icmp_seq=4 ttl=47 time=255 ms
64 bytes from primary.admin.cam.ac.uk (131.111.150.25): icmp_seq=5 ttl=47 time=256 ms
64 bytes from primary.admin.cam.ac.uk (131.111.150.25): icmp_seq=6 ttl=47 time=255 ms
64 bytes from primary.admin.cam.ac.uk (131.111.150.25): icmp_seq=7 ttl=47 time=255 ms
64 bytes from primary.admin.cam.ac.uk (131.111.150.25): icmp_seq=8 ttl=47 time=255 ms
64 bytes from primary.admin.cam.ac.uk (131.111.150.25): icmp_seq=9 ttl=47 time=255 ms
64 bytes from primary.admin.cam.ac.uk (131.111.150.25): icmp_seq=10 ttl=47 time=255 ms

--- www.cam.ac.uk ping statistics ---
10 packets transmitted, 10 received, 0% packet loss, time 9012ms
rtt min/avg/max/mdev = 255.566/255.870/256.284/0.308 ms
\end{lstlisting}

\subsection{}
\paragraph{}Os tempos mínimo, máximo e médio de resposta de uma máquina no endereço \mbox{www.unicamp.br} foram 0,669 ms, 0,788 ms e 0,716 ms respectivamente. O motivo desta grande diferença entre os tempos medidos para a máquina na Universidade de Cambridge, além da distância física que aumenta o tempo de resposta devido ao atraso na propagação do sinal pelos vários meios de interconexão, é o maior número de roteadores pelos quais os pacotes de requisição e resposta devem passar, cada qual com um atraso por causa das filas nas interfaces e no roteamento do pacote.
\begin{lstlisting}
PING cerejeira.unicamp.br (143.106.10.174) 56(84) bytes of data.
64 bytes from cerejeira.unicamp.br (143.106.10.174): icmp_seq=1 ttl=58 time=0.788 ms
64 bytes from cerejeira.unicamp.br (143.106.10.174): icmp_seq=2 ttl=58 time=0.732 ms
64 bytes from cerejeira.unicamp.br (143.106.10.174): icmp_seq=3 ttl=58 time=0.683 ms
64 bytes from cerejeira.unicamp.br (143.106.10.174): icmp_seq=4 ttl=58 time=0.741 ms
64 bytes from cerejeira.unicamp.br (143.106.10.174): icmp_seq=5 ttl=58 time=0.715 ms
64 bytes from cerejeira.unicamp.br (143.106.10.174): icmp_seq=6 ttl=58 time=0.709 ms
64 bytes from cerejeira.unicamp.br (143.106.10.174): icmp_seq=7 ttl=58 time=0.728 ms
64 bytes from cerejeira.unicamp.br (143.106.10.174): icmp_seq=8 ttl=58 time=0.669 ms
64 bytes from cerejeira.unicamp.br (143.106.10.174): icmp_seq=9 ttl=58 time=0.703 ms
64 bytes from cerejeira.unicamp.br (143.106.10.174): icmp_seq=10 ttl=58 time=0.694 ms

--- cerejeira.unicamp.br ping statistics ---
10 packets transmitted, 10 received, 0% packet loss, time 9011ms
rtt min/avg/max/mdev = 0.669/0.716/0.788/0.036 ms
\end{lstlisting}

\subsection{}
\paragraph{}A máquina no endereço www.lrc.ic.unicamp.br está configurada para não responder a \mbox{\emph{ICMP echo resquests}}, simplesmente descartando os pacotes deste tipo que a ela chegam. No teste em laboratório, a máquina no endereço acima respondeu a um pedido HTTP, mas não respondeu ao ping. Este fato demonstra que a ferramenta ping não é confiável para verificação de disponibilidade de uma máquina em rede.



\section{ifconfig}

\paragraph{}O endereço IPv4 da máquina usada no exercício é \mbox{143.106.16.14}; e o endereço IPv6 é \mbox{fe80::7ae7:d1ff:fe55:acf5} associado à interface \emph{em1}. Além disso, há o endereço IPv4 \mbox{127.0.0.1} associado à interface \emph{lo}. A interface \emph{em1} enviou 76581602 bytes e recebeu 1970402164 bytes; enquanto a interface \emph{lo} enviou 11964 bytes e recebeu 11964 bytes.
\begin{lstlisting}
em1: flags=4163<UP,BROADCAST,RUNNING,MULTICAST>  mtu 1500
        inet 143.106.16.14  netmask 255.255.255.192  broadcast 143.106.16.63
        inet6 fe80::7ae7:d1ff:fe55:acf5  prefixlen 64  scopeid 0x20<link>
        ether 78:e7:d1:55:ac:f5  txqueuelen 1000  (Ethernet)
        RX packets 1329404  bytes 1970402164 (1.8 GiB)
        RX errors 0  dropped 0  overruns 0  frame 0
        TX packets 663584  bytes 76581602 (73.0 MiB)
        TX errors 0  dropped 0 overruns 0  carrier 0  collisions 0
        device interrupt 19  memory 0xf0500000-f0520000  

lo: flags=73<UP,LOOPBACK,RUNNING>  mtu 65536
        inet 127.0.0.1  netmask 255.0.0.0
        inet6 ::1  prefixlen 128  scopeid 0x10<host>
        loop  txqueuelen 0  (Local Loopback)
        RX packets 150  bytes 11964 (11.6 KiB)
        RX errors 0  dropped 0  overruns 0  frame 0
        TX packets 150  bytes 11964 (11.6 KiB)
        TX errors 0  dropped 0 overruns 0  carrier 0  collisions 0
\end{lstlisting}



\section{ifconfig}

\paragraph{}Inicialmente, interface \emph{lo} havia enviado e recebido 150 pacotes. Após o envio de dois pacotes \emph{ICMP echo request} para esta interface, nota-se que tanto os pacotes enviados quanto os recebidos foram acrescidos de quatro. O que ocorreu foi que foram enviados dois pacotes \emph{ICMP echo request} para o endereço \mbox{127.0.0.1} na interface \emph{lo} e esta mesma interface recebeu os dois pacotes. A máquina local, então, enviou dois pacotes com a resposta para a interface que recebeu os pedidos (\emph{lo}) e a mesma interface recebeu as respostas. Assim, temos quatro pacotes a mais enviados e recebidos.
\paragraph{}Esta interface é um laço para a própria máquina---um endereço de \emph{loopback}---e é chamado de \emph{localhost}.



\section{route}

\paragraph{}A tabela de roteamento da máquina de teste inclui a rota padrão para todos os endereços, cuja a saída é a interface \emph{em1} e tem o gateway em \mbox{143.106.16.62}. Além disso, há uma sub-rede de 62 hosts (de \mbox{143.106.16.1} a \mbox{143.106.16.62}) definida.
\begin{lstlisting}
Kernel IP routing table
Destination     Gateway         Genmask         Flags Metric Ref    Use Iface
default         143.106.16.62   0.0.0.0         UG    0      0        0 em1
143.106.16.0    0.0.0.0         255.255.255.192 U     0      0        0 em1
link-local      0.0.0.0         255.255.0.0     U     1002   0        0 em1
\end{lstlisting}



\section{nslookup}

\subsection{}
\paragraph{}O domínio \mbox{www.google.com} está configurado com os seguintes endereços IP: \mbox{173.194.42.176}, \mbox{173.194.42.177}, \mbox{173.194.42.178}, \mbox{173.194.42.179} e \mbox{173.194.42.180}. Ter vários endereços IP para o mesmo domínio aumenta a disponibilidade do serviço, pois, caso um endereço esteja indisponível ou lento, um segundo endereço pode ser usado, servindo como uma espécie de \emph{load balancing}.
\paragraph{}O endereço IP do servidor DNS da máquina de teste é \mbox{1143.106.16.144}, cujo domínio é \mbox{monica.lab.ic.unicamp.br}.
\begin{lstlisting}
Server:		143.106.16.144
Address:	143.106.16.144#53

Non-authoritative answer:
Name:	www.google.com
Address: 173.194.42.176
Name:	www.google.com
Address: 173.194.42.177
Name:	www.google.com
Address: 173.194.42.178
Name:	www.google.com
Address: 173.194.42.179
Name:	www.google.com
Address: 173.194.42.180
\end{lstlisting}
\begin{lstlisting}
Server:		143.106.16.144
Address:	143.106.16.144#53

144.16.106.143.in-addr.arpa	name = monica.lab.ic.unicamp.br.
\end{lstlisting}


\subsection{}
\paragraph{}O nome relacionado ao endereço \mbox{127.0.0.1} é \emph{localhost} e este é o nome dado a máquina local.
\begin{lstlisting}
Server:		143.106.16.144
Address:	143.106.16.144#53

1.0.0.127.in-addr.arpa	name = localhost.
\end{lstlisting}



\section{traceroute}

\subsection{}
\paragraph{}São dez roteadores entre a máquina de teste e o host \mbox{www.google.com} e todos os hosts com nome de domínio estão localizados no Brasil.
\begin{lstlisting}
traceroute to www.google.com (173.194.42.176), 30 hops max, 60 byte packets
 1  * * *
 2  143.106.16.150 (143.106.16.150)  0.160 ms  0.159 ms  0.153 ms
 3  143.106.7.129 (143.106.7.129)  0.419 ms  0.414 ms  0.401 ms
 4  area3-gw.unicamp.br (143.106.1.129)  3.481 ms  3.477 ms  3.741 ms
 5  ptp-nct-nbs.unicamp.br (143.106.199.13)  0.362 ms  0.639 ms  0.634 ms
 6  as15169-s2.sp.ptt.br (187.16.218.58)  4.179 ms  4.107 ms  4.089 ms
 7  209.85.254.136 (209.85.254.136)  4.415 ms  4.406 ms  4.401 ms
 8  209.85.245.53 (209.85.245.53)  14.399 ms  17.152 ms 209.85.246.5 (209.85.246.5)  16.902 ms
 9  209.85.253.171 (209.85.253.171)  17.548 ms  17.729 ms  14.430 ms
10  rio01s06-in-f16.1e100.net (173.194.42.176)  11.065 ms  11.012 ms  11.006 ms
\end{lstlisting}

\subsection{}
\paragraph{}Entre a máquina de teste e o host em \mbox{www.cam.ac.uk} há dezessete roteadores e os cinco primeiros são comuns à rota até o host em \mbox{www.google.com}.
\begin{lstlisting}
traceroute to www.cam.ac.uk (131.111.150.25), 30 hops max, 60 byte packets
 1  * * *
 2  143.106.16.150 (143.106.16.150)  0.291 ms  0.287 ms  0.280 ms
 3  143.106.7.129 (143.106.7.129)  0.527 ms  0.525 ms  0.519 ms
 4  area3-gw.unicamp.br (143.106.1.129)  0.512 ms  0.729 ms  0.727 ms
 5  ptp-nct-nbs.unicamp.br (143.106.199.13)  26.505 ms ptp-ncc-nbs.unicamp.br (143.106.199.9)  4.141 ms ptp-nct-nbs.unicamp.br (143.106.199.13)  26.499 ms
 6  * * *
 7  rnp-br-spau.core.redclara.net (200.0.204.129)  3.874 ms  3.918 ms  3.913 ms
 8  RedCLARA.mx1.mad.es.geant.net (62.40.124.137)  227.107 ms  227.118 ms  226.923 ms
 9  ae3.mx1.par.fr.geant.net (62.40.98.65)  244.397 ms  244.425 ms  244.419 ms
10  ae1.mx1.lon.uk.geant.net (62.40.98.76)  250.253 ms  250.254 ms  250.199 ms
11  janet-gw.mx1.lon.uk.geant.net (62.40.124.198)  250.198 ms  250.391 ms  250.369 ms
12  ae28.lowdss-sbr1.ja.net (146.97.33.18)  253.632 ms  254.893 ms  253.657 ms
13  ae0.camb-rbr2.ja.net (146.97.37.186)  255.849 ms  255.714 ms  255.702 ms
14  University-of-Cambridge.Camb-rbr1.eastern.ja.net (146.97.130.2)  255.973 ms  255.956 ms  256.034 ms
15  route-enet.route-mill.net.cam.ac.uk (192.84.5.93)  256.031 ms  255.883 ms  255.845 ms
16  route-mill.route-nwest.net.cam.ac.uk (192.84.5.138)  256.076 ms  255.969 ms  255.988 ms
17  primary.admin.cam.ac.uk (131.111.150.25)  255.982 ms  256.163 ms  256.209 ms
\end{lstlisting}

\subsection{}
\paragraph{}Não é possível determinar com exatidão quantos são os roteadores entre a máquina de teste e o host \mbox{home.pl}, pois há um firewall ou NAT router no gateway desta rede. No entanto, pode-se dizer que há pelo menos 21 roteadores entre estas duas máquinas.
\begin{lstlisting}
traceroute to home.pl (212.85.96.1), 60 hops max, 60 byte packets
 1  * * *
 2  143.106.16.150 (143.106.16.150)  0.187 ms  0.184 ms  0.175 ms
 3  143.106.7.129 (143.106.7.129)  0.453 ms  0.448 ms  0.442 ms
 4  area3-gw.unicamp.br (143.106.1.129)  0.695 ms  0.695 ms  0.972 ms
 5  ptp-ncc-nbs.unicamp.br (143.106.199.9)  0.398 ms  0.662 ms ptp-nct-nbs.unicamp.br (143.106.199.13)  0.381 ms
 6  * * *
 7  200.143.252.38 (200.143.252.38)  110.245 ms  109.993 ms  109.985 ms
 8  xe-9-3-2.edge2.Miami2.Level3.net (4.59.242.41)  109.948 ms  110.126 ms  110.122 ms
 9  ae-32-52.ebr2.Miami1.Level3.net (4.69.138.123)  253.993 ms  253.984 ms  253.966 ms
10  ae-2-2.ebr2.Atlanta2.Level3.net (4.69.140.142)  252.845 ms  252.703 ms  252.687 ms
11  * * *
12  ae-2-2.ebr1.Washington1.Level3.net (4.69.132.86)  252.852 ms  253.152 ms  252.652 ms
13  ae-81-81.csw3.Washington1.Level3.net (4.69.134.138)  252.634 ms ae-71-71.csw2.Washington1.Level3.net (4.69.134.134)  257.706 ms ae-61-61.csw1.Washington1.Level3.net (4.69.134.130)  254.445 ms
14  ae-92-92.ebr2.Washington1.Level3.net (4.69.134.157)  253.523 ms ae-82-82.ebr2.Washington1.Level3.net (4.69.134.153)  251.970 ms  251.947 ms
15  ae-43-43.ebr2.Paris1.Level3.net (4.69.137.57)  252.591 ms ae-41-41.ebr2.Paris1.Level3.net (4.69.137.49)  252.808 ms ae-42-42.ebr2.Paris1.Level3.net (4.69.137.53)  252.667 ms
16  ae-47-47.ebr1.Frankfurt1.Level3.net (4.69.143.141)  251.435 ms  252.666 ms ae-46-46.ebr1.Frankfurt1.Level3.net (4.69.143.137)  252.960 ms
17  ae-71-71.csw2.Frankfurt1.Level3.net (4.69.140.6)  252.890 ms ae-61-61.csw1.Frankfurt1.Level3.net (4.69.140.2)  255.422 ms ae-71-71.csw2.Frankfurt1.Level3.net (4.69.140.6)  255.204 ms
18  ae-64-64.ebr4.Frankfurt1.Level3.net (4.69.163.17)  253.586 ms  252.998 ms ae-74-74.ebr4.Frankfurt1.Level3.net (4.69.163.21)  251.487 ms
19  ae-1-9.bar1.Warsaw1.Level3.net (4.69.153.70)  251.896 ms  253.093 ms  251.606 ms
20  LWLcom-Bremen.level3.net (213.242.117.58)  253.271 ms  253.618 ms  254.204 ms
21  xe-2-3.gate2.home.net.pl (62.129.251.190)  253.981 ms  254.275 ms  253.532 ms
22  * * *
23  * * *
24  * * *
\end{lstlisting}
\paragraph{}A rota contrária não é a mesma, mas ela tem alguns trechos em comum: a rota dentro da rede da Unicam e um trecho no \emph{backbone} da Level3.
\begin{lstlisting}
 HOST: vmy1.home.net.pl            Loss%   Snt   Last   Avg  Best  Wrst StDev
  1.|-- adx01.home.net.pl          0.0%     5    0.3   0.3   0.3   0.4   0.0
  2.|-- 62.129.251.154             0.0%     5    0.4   0.4   0.3   0.5   0.1
  3.|-- dialup-212.162.18.57.fran  0.0%     5    0.4   0.5   0.4   0.5   0.0
  4.|-- ae-9-9.ebr4.Frankfurt1.Le  0.0%     5  145.4 143.8 143.3 145.4   0.9
  5.|-- ae-64-64.csw1.Frankfurt1.  0.0%     5  143.1 143.1 143.0 143.3   0.1
  6.|-- ae-61-61.ebr1.Frankfurt1.  0.0%     5  142.4 142.4 142.4 142.4   0.0
  7.|-- ae-45-45.ebr2.Paris1.Leve  0.0%     5  141.5 141.5 141.5 141.6   0.1
  8.|-- ae-43-43.ebr2.Washington1  0.0%     5  144.3 144.3 144.3 144.4   0.0
  9.|-- ae-82-82.csw3.Washington1  0.0%     5  145.4 145.8 145.4 147.2   0.8
 10.|-- ae-81-81.ebr1.Washington1  0.0%     5  143.2 143.3 143.2 143.3   0.1
 11.|-- ???                       100.0     5    0.0   0.0   0.0   0.0   0.0
 12.|-- ae-102-102.ebr2.Atlanta2.  0.0%     5  144.4 144.4 144.4 144.5   0.0
 13.|-- ae-2-2.ebr2.Miami1.Level3  0.0%     5  145.5 144.5 144.3 145.5   0.6
 14.|-- ae-2-52.edge2.Miami2.Leve  0.0%     5  143.2 143.3 143.2 143.5   0.1
 15.|-- LATIN-AMERI.edge2.Miami2.  0.0%     5  143.7 143.7 143.6 143.8   0.1
 16.|-- 200.143.252.37             0.0%     5  252.0 251.8 251.3 253.1   0.8
 17.|-- rnp-ncc.unicamp.br         0.0%     5  252.7 252.7 252.7 252.8   0.0
 18.|-- ptp-nbs-ncc.unicamp.br    20.0%     5  252.8 252.8 252.8 252.8   0.0
 19.|-- ic-gw.unicamp.br          20.0%     5  255.9 255.2 253.1 255.9   1.4
 20.|-- ic3-gw.ic.unicamp.br      20.0%     5  254.3 254.2 254.1 254.3   0.1
 21.|-- ???                       100.0     5    0.0   0.0   0.0   0.0   0.0
 22.|-- faith.lab.ic.unicamp.br   20.0%     5  254.8 254.7 254.7 254.8   0.0
\end{lstlisting}

\subsection{}
\paragraph{}A partir do sétimo \emph{hop} a latência aumenta de uma ordem de grandeza e o oitavo \emph{hop} tem um nome contendo ``Miami'' e uma latência semelhante, o sétimo \emph{hop} deve ser um enlace transatlântico.



\section{netstat}

\subsection{}
\paragraph{}O programa netstat fornece as seguintes informações sobre cada uma das conexões TCP da máquina de teste: o tamanho das filas de recepção e transmissão nos \emph{sockets}, o endereço local e o endereço remoto (com as portas usadas em cada conexão) e o estado da conexão. \paragraph{}Para a requisição da página inicial do website em \mbox{www.unicamp.br} a máquina de teste estabeleceu pelo menos seis conexões com a máquina \mbox{cerejeira.unicamp.br} na porta 80 (HTTP). Algumas das conexões podem ter sido encerradas entre a requisição da página pelo \emph{browser} e a execução do comando netstat. Das seis conexões, todas estão no estado \emph{established}, que significa que estão ou transmitindo ou abertas esperando uma transmissão.

\subsection{}
\paragraph{}Há uma conexão com a máquina \mbox{cebolinha.ic.unicamp.br} usando a porta para o Network File System (NFS).

\subsection{}
\paragraph{}Para acessar servidores HTTP, a máquina de teste utiliza de portas com numeração aleatória e alta, bem acima de 1023. Entre conexões estabelecidas entre um mesmo website, a numeração das portas é bem próxima, sendo algumas vezes sequencial.
\begin{lstlisting}
Active Internet connections (w/o servers)
Proto Recv-Q Send-Q Local Address           Foreign Address         State      
tcp        0      0 faith.lab.ic.unicamp.br:34031 cerejeira.unicamp.br:http ESTABLISHED
tcp        0      0 faith.lab.ic.unicamp.br:791 cebolinha.lab.ic.unicamp.br:nfs ESTABLISHED
tcp        0      0 faith.lab.ic.unicamp.br:34678 www.nearlyfreespeech.net:https ESTABLISHED
tcp        0      0 faith.lab.ic.unicamp.br:45450 wfe0.ysv.freebsd.org:http ESTABLISHED
tcp        0      0 faith.lab.ic.unicamp.br:45455 wfe0.ysv.freebsd.org:http ESTABLISHED
tcp        0      0 faith.lab.ic.unicamp.br:45454 wfe0.ysv.freebsd.org:http ESTABLISHED
tcp        0      0 faith.lab.ic.unicamp.br:32947 ec2-54-232-107-27.sa-east-1.compute.amazonaws.com:https ESTABLISHED
tcp        0      0 faith.lab.ic.unicamp.br:37726 server-54-230-224-195.gig50.r.cloudfront.net:https ESTABLISHED
tcp        0      0 faith.lab.ic.unicamp.br:34032 cerejeira.unicamp.br:http ESTABLISHED
tcp        0      0 faith.lab.ic.unicamp.br:32945 ec2-54-232-107-27.sa-east-1.compute.amazonaws.com:https ESTABLISHED
tcp        0      0 faith.lab.ic.unicamp.br:55429 vhost.phx7.nearlyfreespeech.net:http ESTABLISHED
tcp        1      0 faith.lab.ic.unicamp.br:38905 6bone.informatik.uni-leipzig.de:http CLOSE_WAIT 
tcp        0      0 faith.lab.ic.unicamp.br:38503 wildebeest.gnu.org:http ESTABLISHED
tcp        0      0 faith.lab.ic.unicamp.br:46927 vhost.phx8.nearlyfreespeech.net:http ESTABLISHED
tcp        0      0 faith.lab.ic.unicamp.br:40084 yk-in-f95.1e100.net:https ESTABLISHED
tcp        0      0 faith.lab.ic.unicamp.br:50689 vhost.phx4.nearlyfreespeech.net:http ESTABLISHED
tcp        0      0 faith.lab.ic.unicamp.br:55899 gru06s12-in-f6.1e100.net:http ESTABLISHED
tcp        1      0 faith.lab.ic.unicamp.br:51075 traceroute.org:http     CLOSE_WAIT 
tcp        0      0 faith.lab.ic.unicamp.br:50510 gudrun.archlinux.org:https CLOSE_WAIT 
tcp        0      0 faith.lab.ic.unicamp.br:37727 server-54-230-224-195.gig50.r.cloudfront.net:https ESTABLISHED
tcp        0      0 faith.lab.ic.unicamp.br:55428 vhost.phx7.nearlyfreespeech.net:http ESTABLISHED
tcp        0      0 faith.lab.ic.unicamp.br:50669 vhost.phx4.nearlyfreespeech.net:http ESTABLISHED
tcp        0      0 faith.lab.ic.unicamp.br:32946 ec2-54-232-107-27.sa-east-1.compute.amazonaws.com:https ESTABLISHED
tcp        0      0 faith.lab.ic.unicamp.br:45453 wfe0.ysv.freebsd.org:http ESTABLISHED
tcp        1      0 faith.lab.ic.unicamp.br:34836 mozdev.mozdev.org:http  CLOSE_WAIT 
tcp        1      0 faith.lab.ic.unicamp.br:51072 traceroute.org:http     CLOSE_WAIT 
tcp        0      0 faith.lab.ic.unicamp.br:38502 wildebeest.gnu.org:http ESTABLISHED
tcp        0      0 faith.lab.ic.unicamp.br:45452 wfe0.ysv.freebsd.org:http ESTABLISHED
tcp        0      0 faith.lab.ic.unicamp.br:38506 wildebeest.gnu.org:http ESTABLISHED
tcp        0      0 faith.lab.ic.unicamp.br:46928 vhost.phx8.nearlyfreespeech.net:http ESTABLISHED
tcp        0      0 faith.lab.ic.unicamp.br:34029 cerejeira.unicamp.br:http ESTABLISHED
tcp        0      0 faith.lab.ic.unicamp.br:37724 server-54-230-224-195.gig50.r.cloudfront.net:https ESTABLISHED
tcp        0      0 faith.lab.ic.unicamp.br:39236 213.175.193.206:9940    ESTABLISHED
tcp        0      0 faith.lab.ic.unicamp.br:37752 server-54-230-224-195.gig50.r.cloudfront.net:https ESTABLISHED
tcp        0      0 faith.lab.ic.unicamp.br:34030 cerejeira.unicamp.br:http ESTABLISHED
tcp        0      0 faith.lab.ic.unicamp.br:43646 mingau.lab.ic.unicamp.br:ldap ESTABLISHED
tcp        0      0 faith.lab.ic.unicamp.br:34028 cerejeira.unicamp.br:http ESTABLISHED
tcp        0      0 faith.lab.ic.unicamp.br:37728 server-54-230-224-195.gig50.r.cloudfront.net:https ESTABLISHED
tcp        0      0 faith.lab.ic.unicamp.br:38504 wildebeest.gnu.org:http ESTABLISHED
tcp        0      0 faith.lab.ic.unicamp.br:exp1 cebolinha.lab.ic.unicamp.br:45879 ESTABLISHED
tcp        0      0 faith.lab.ic.unicamp.br:37725 server-54-230-224-195.gig50.r.cloudfront.net:https ESTABLISHED
tcp        0      0 faith.lab.ic.unicamp.br:45451 wfe0.ysv.freebsd.org:http ESTABLISHED
tcp        0      0 faith.lab.ic.unicamp.br:39705 50.31.164.192:https     ESTABLISHED
tcp        0      0 faith.lab.ic.unicamp.br:34027 cerejeira.unicamp.br:http ESTABLISHED
tcp        1      0 faith.lab.ic.unicamp.br:51074 traceroute.org:http     CLOSE_WAIT 
tcp        0      0 faith.lab.ic.unicamp.br:38505 wildebeest.gnu.org:http ESTABLISHED
tcp        0      0 faith.lab.ic.unicamp.br:38501 wildebeest.gnu.org:http ESTABLISHED
tcp        1      0 faith.lab.ic.unicamp.br:51061 traceroute.org:http     CLOSE_WAIT 
tcp        0      0 faith.lab.ic.unicamp.br:39688 50.31.164.192:https     ESTABLISHED
tcp        0      0 faith.lab.ic.unicamp.br:940 franjinha.lab.ic.unicamp.br:nfs ESTABLISHED
tcp        0      0 faith.lab.ic.unicamp.br:39706 50.31.164.192:https     ESTABLISHED
\end{lstlisting}



\section{telnet}

\subsection{}
\paragraph{}É possível conectar-se por telnet a um servidor HTTP, pois este não requer autentificação. O comando usado pode ser \verb|telnet www.google.com 80| ou somente telnet e depois usar os comandos internos (open) para iniciar a conexão.
\begin{lstlisting}
bash-4.2$ telnet www.google.com 80
Trying 173.194.42.177...
Connected to www.google.com.
Escape character is '^]'.
GET /
HTTP/1.0 302 Found
Cache-Control: private
Content-Type: text/html; charset=UTF-8
Location: http://www.google.com.br/?gfe_rd=cr&ei=81IPU8ziOomC8Qevv4CACQ
Content-Length: 262
Date: Thu, 27 Feb 2014 15:00:03 GMT
Server: GFE/2.0
Alternate-Protocol: 80:quic

<HTML><HEAD><meta http-equiv="content-type" content="text/html;charset=utf-8">
<TITLE>302 Moved</TITLE></HEAD><BODY>
<H1>302 Moved</H1>
The document has moved
<A HREF="http://www.google.com.br/?gfe_rd=cr&amp;ei=81IPU8ziOomC8Qevv4CACQ">here</A>.
</BODY></HTML>
Connection closed by foreign host.
\end{lstlisting}

\subsection{}Não é possível conectar-se a uma porta do host sem que exista um servidor---uma aplicação para o qual a conexão a uma determinada porta é encaminhada---``escutando-a''. Na máquina local não há um servidor HTTP configurado para ouvir a porta 80, portanto, a conexão por telnet falha.
\begin{lstlisting}
bash-4.2$ telnet
telnet> open 127.0.0.1 80
Trying 127.0.0.1...
telnet: connect to address 127.0.0.1: Connection refused
telnet> quit
\end{lstlisting}



\end{document}

