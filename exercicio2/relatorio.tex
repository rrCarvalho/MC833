\documentclass[a4paper,10pt,oneside,final,titlepage,onecolumn]{article}

\usepackage{ucs}
\usepackage[portuguese]{babel}
\usepackage[utf8x]{inputenc}
\usepackage[T1]{fontenc}
\usepackage{textcomp}

\usepackage{listings}
\usepackage{color}

\definecolor{dkgreen}{rgb}{0,0.6,0}
\definecolor{gray}{rgb}{0.5,0.5,0.5}
\definecolor{mauve}{rgb}{0.58,0,0.82}

\lstset{frame=tb,
  language=bash,
  aboveskip=3mm,
  belowskip=3mm,
  showstringspaces=false,
  columns=flexible,
  basicstyle={\scriptsize\ttfamily},
  numbers=none,  
  breaklines=true,
  breakatwhitespace=true
  tabsize=3
}



\title{Exercício 2 de MC833 --- Programação em Redes de Computadores}
\author{Raul Rabelo Carvalho, 105607, turma A}



\begin{document}



\maketitle



\section{Quais são as interfaces nas quais o tcpdump pode escutar/capturar dados? Essas interfaces são as mesmas mostradas pelo comando ifconfig?}
\paragraph{}Para verificar quais interfaces de rede estão disponíveis para captura de pacotes ao \verb|tcpdump|, deve-se usar o comando \verb|tcpdump -D|. Na máquina sabbath da sala CC302 nenhuma interface está disponível para captura de pacotes, pois os alunos não podem executar comandos como root. O comando \verb|ifconfig -a|, no entanto, lista duas interfaces: lo e em1, como mostrado abaixo.
\paragraph{}Em um laptop conectado a um \emph{NAT router} de uma rede doméstica pela interface sem fio, temos para o comando \verb|sudo tcpdump -D| sete interfaces listadas, como mostrado abaixo.
\begin{lstlisting}
rrcarvalho@Tarkin ~/MC823/exercicio2 $ sudo tcpdump -D
1.nflog (Linux netfilter log (NFLOG) interface)
2.nfqueue (Linux netfilter queue (NFQUEUE) interface)
3.dbus-system (D-Bus system bus)
4.enp2s0
5.wlp3s0
6.any (Pseudo-device that captures on all interfaces)
7.lo
\end{lstlisting}
\paragraph{}As duas primeiras linhas da saída mostram \emph{kernel hooks} do projeto Netfilter\footnote{http://netfilter.org/} para filtragem de pacotes no Linux; a terceira linha é um \emph{Unix socket} usado para comunicação entre processos. A partir da quarta linha temos as interfaces de rede; sendo \verb|enp2s0| a interface Ethernet do laptop, enquando wlp3s0 é a interface sem fio; na sétima linha é mostrado a interface associada ao localhost da máquina.
\paragraph{}O comando \verb|ifconfig -a|, no entando, lista somente três interfaces:
\begin{lstlisting}
rrcarvalho@Tarkin ~/MC823/exercicio2 $ ifconfig -a
enp2s0: flags=4099<UP,BROADCAST,MULTICAST>  mtu 1500
        ether dc:0e:a1:c7:9f:29  txqueuelen 1000  (Ethernet)
        RX packets 0  bytes 0 (0.0 B)
        RX errors 0  dropped 0  overruns 0  frame 0
        TX packets 0  bytes 0 (0.0 B)
        TX errors 0  dropped 0 overruns 0  carrier 0  collisions 0

lo: flags=73<UP,LOOPBACK,RUNNING>  mtu 65536
        inet 127.0.0.1  netmask 255.0.0.0
        inet6 ::1  prefixlen 128  scopeid 0x10<host>
        loop  txqueuelen 0  (Local Loopback)
        RX packets 434  bytes 33476 (32.6 KiB)
        RX errors 0  dropped 0  overruns 0  frame 0
        TX packets 434  bytes 33476 (32.6 KiB)
        TX errors 0  dropped 0 overruns 0  carrier 0  collisions 0

wlp3s0: flags=4163<UP,BROADCAST,RUNNING,MULTICAST>  mtu 1500
        inet 10.17.23.103  netmask 255.255.255.0  broadcast 10.17.23.255
        inet6 fe80::5ec9:d3ff:fe14:2f4a  prefixlen 64  scopeid 0x20<link>
        ether 5c:c9:d3:14:2f:4a  txqueuelen 1000  (Ethernet)
        RX packets 9995  bytes 3852001 (3.6 MiB)
        RX errors 0  dropped 1299  overruns 0  frame 0
        TX packets 1437  bytes 199562 (194.8 KiB)
        TX errors 0  dropped 0 overruns 0  carrier 0  collisions 0
\end{lstlisting}
\paragraph{}As interfaces, neste caso, são todas de rede e são a interface Ethernet (enp2s0), o localhost (lo) e a interface sem fio (wlp3s0).



\section{Qual é o endereço IP do nós maple e willow na rede em questão? }
\paragraph{}Para descobrir os endereços IP das duas máquinas da comunicação registrada em \verb|tcpdump.dat|, é possível utilizar a opção \verb|-n| do \verb|tcpdump|. Deve-se executar o comando \verb|tcpdump -r tcpdump.dat > outfile.txt| e procurar uma linha com os nomes para os quais se quer o endereço IP; no caso a terceira linha contém os dois nomes:
\begin{lstlisting}
01:34:41.473518 IP willow.csail.mit.edu.39675 > maple.csail.mit.edu.commplex-link: Flags [S], seq 1258159963, win 14600, options [mss 1460,sackOK,TS val 282136473 ecr 0,nop,wscale 7], length 0
\end{lstlisting}
\paragraph{}Em seguida, executa-se \verb|tcpdump -r tcpdump.dat -n > enderecos.txt| e verifica-se a mesma terceira linha:
\begin{lstlisting}
01:34:41.473518 IP 128.30.4.222.39675 > 128.30.4.223.5001: Flags [S], seq 1258159963, win 14600, options [mss 1460,sackOK,TS val 282136473 ecr 0,nop,wscale 7], length 0
\end{lstlisting}
\paragraph{}Temos, então, que a máquina com o nome willow está no endereço IP \verb|128.30.4.222| e a máquina maple está em \verb|128.30.4.223|. Note que a mesma informação poderia ser obtida usando o programa \verb|nslookup|.



\section{Qual é o endereço MAC dos nós maple e willow? }
\paragraph{}Ainda que seja possível se obter os endereços IP das máquinas willow e maple consultando um servidor DNS, endereços MAC não estão disponíveis externamente à rede Ethernet local. Ainda sim, o registro \verb|tcpdump.dat| contém os endereços MAC de cada uma das máquinas de origem e destino dos pacotes capturados pelo \verb|tcpdump|. A opção \verb|-e| exibe estas informações.
\paragraph{}Seria possível obter o endereço MAC na terceira linha novamente, mas será empregado um filtro para listar somente os pacotes ARP, que é o protocolo de manutenção da rede Ethernet.
\begin{lstlisting}
rrcarvalho@Tarkin ~/MC823/exercicio2 $ tcpdump -r tcpdump.dat -e arp
reading from file tcpdump.dat, link-type EN10MB (Ethernet)
01:34:41.473036 00:16:ea:8e:28:44 (oui Unknown) > Broadcast, ethertype ARP (0x0806), length 42: Request who-has maple.csail.mit.edu tell willow.csail.mit.edu, length 28
01:34:41.473505 00:16:ea:8d:e5:8a (oui Unknown) > 00:16:ea:8e:28:44 (oui Unknown), ethertype ARP (0x0806), length 42: Reply maple.csail.mit.edu is-at 00:16:ea:8d:e5:8a (oui Unknown), length 28
\end{lstlisting}
\paragraph{}No primeiro pacote, temos uma máquina no endereço MAC \verb|00:16:ea:8e:28:44| requisitando o endereço físico da máquina maple. Como a comunicação capturada ocorre somente entre as máquinas maple e willow, podemos assumir que este endereço MAC é da máquina willow. Mas essa informação pode ser confirmada observando o terceiro pacote mais adiante.
\paragraph{}Já no segundo pacote, alguma máquina responde que maple está no endereço MAC \verb|00:16:ea:8d:e5:8a|; no caso, a própria maple foi a máquina que respondeu ao pedido da willow.
\paragraph{}Para verificar o endereço MAC da máquina willow, observa-se no terceiro pacote listado pelo comando \verb|tcpdump -r tcpdump.dat -e > ethernet.txt| que \verb|00:16:ea:8e:28:44| é de fato o endereço de willow.
\begin{lstlisting}
01:34:41.473518 00:16:ea:8e:28:44 (oui Unknown) > 00:16:ea:8d:e5:8a (oui Unknown), ethertype IPv4 (0x0800), length 74: willow.csail.mit.edu.39675 > maple.csail.mit.edu.commplex-link: Flags [S], seq 1258159963, win 14600, options [mss 1460,sackOK,TS val 282136473 ecr 0,nop,wscale 7], length 0
\end{lstlisting}



\section{Qual é a porta TCP usada pelos nós maple e willow? Qual é o tipo de porta que está sendo utilizada pela fonte nesta conexão?}
\paragraph{}Usando novamente a opção \verb|-n|, pode-se obter o número das portas utilizadas por maple e willow. Sem esta opção, mesmo algumas portas acima de 1024 são associadas a uma aplicação oficial e assim mostradas. No caso específico da conexão estudada, a porta de destino da conexão (na máquina maple) é 5001, registrada na IANA como \verb|commplex-link|, mas que, no entanto, pode não ser a aplicação usada nesta conexão, já que esta é uma porta associada a múltiplas aplicações não oficiais.
\begin{lstlisting}
01:34:41.473518 IP 128.30.4.222.39675 > 128.30.4.223.5001: Flags [S], seq 1258159963, win 14600, options [mss 1460,sackOK,TS val 282136473 ecr 0,nop,wscale 7], length 0
\end{lstlisting}
\paragraph{}A porta na máquina willow, a qual originou a conexão, é 39675. Esta é uma porta bem acima das \emph{well-known ports}, escolhida aleatoriamente pela máquina para identificar unicamente esta conexão; esta porta não está associada a nenhuma aplicação específica.



\section{Quantos kilobytes foram transferidos durante essa sessão TCP? Qual foi a duração da sessão? Baseado nas respostas anteriores, qual é a vazão (em Kilobytes/segundos) do fluxo TCP entre willow e maple?}
\paragraph{}Para se descobrir quantos bytes foram enviados na conexão, nota-se que o \verb|tcpdump| usa uma numeração relativa para exibir o campo SEQ do cabeçalho TCP. O primeiro pacote de push da conexão é:
\begin{lstlisting}
01:34:41.474166 IP willow.39675 > maple.commplex-link: Flags [P.], seq 1:25, ack 1, win 115, options [nop,nop,TS val 282136474 ecr 282202089], length 24
\end{lstlisting}
\paragraph{}Foi usado o comando \verb|tcpdump -r tcpdump.dat -N > short.txt| para se obter a lista de pacotes de onde a linha acima foi retirada.
\paragraph{}Assim, basta localizar o último pacote ACK reconhecendo o recebimento de um intervalo de bytes. O último pacote ACK desta sequência é:
\begin{lstlisting}
01:34:44.329956 IP maple.commplex-link > willow.39675: Flags [.], ack 1572890, win 820, options [nop,nop,TS val 282204945 ecr 282139320], length 0
\end{lstlisting}
que reconhece o recebimento do pacote:
\begin{lstlisting}
01:34:44.311921 IP willow.39675 > maple.commplex-link: Flags [FP.], seq 1572017:1572889, ack 1, win 115, options [nop,nop,TS val 282139311 ecr 282204927], length 872
\end{lstlisting}
que é o pacote com as flags FIN e PUSH ativadas.
\paragraph{}Portanto, pode-se afirmar que foram enviados 1 572 889 bytes em um intervalo de 2,865497 segundos. O tempo foi calculado desde o início do \emph{three-way handshake} até que willow reconheça o encerramento da sessão; os dois pacotes referentes a este cálculo estão listados abaixo.
\begin{lstlisting}
01:34:41.473518 IP willow.39675 > maple.commplex-link: Flags [S], seq 1258159963, win 14600, options [mss 1460,sackOK,TS val 282136473 ecr 0,nop,wscale 7], length 0
.
.
.
01:34:44.339015 IP willow.39675 > maple.commplex-link: Flags [.], ack 2, win 115, options [nop,nop,TS val 282139339 ecr 282204955], length 0
\end{lstlisting}
\paragraph{}A taxa de transferência foi, então, de aproximadamente \verb|548,9062 kB/s|.



\section{Qual é o round-trip time (RTT), em segundos, entre willow e maple baseado no pacote 1473:2921 e seu acknowledgment? Veja o arquivo outfile.txt e encontre o RTT do pacote 13057:14505. Porque esses dois valores são diferentes?}
\paragraph{}Para o cálculo do RTT do pacote 1473:2921, foram usados os pacotes:
\begin{lstlisting}
01:34:41.474225 IP willow.39675 > maple.commplex-link: Flags [.], seq 1473:2921, ack 1, win 115, options [nop,nop,TS val 282136474 ecr 282202089], length 1448
.
.
.
01:34:41.482047 IP maple.commplex-link > willow.39675: Flags [.], ack 2921, win 159, options [nop,nop,TS val 282202095 ecr 282136474], length 0
\end{lstlisting}
O RTT calculado para estes pacotes foi de \verb|RTT1 = 7,822 ms|.
\paragraph{}Para o cálculo do RTT do pacote 13057:14505, foram usados os pacotes:
\begin{lstlisting}
01:34:41.474992 IP willow.39675 > maple.commplex-link: Flags [.], seq 13057:14505, ack 1, win 115, options [nop,nop,TS val 282136474 ecr 282202090], length 1448
.
.
.
01:34:41.499373 IP maple.commplex-link > willow.39675: Flags [.], ack 14505, win 331, options [nop,nop,TS val 282202114 ecr 282136474], length 0
\end{lstlisting}
O RTT, neste caso, foi \verb|RTT2 = 4.381 ms|.
\paragraph{}A diferença entre estes valores se deve ao atraso da máquina maple em começar a enviar os pacotes ACK referentes aos bytes transmitidos.



\section{Identifique os procedimentos three-way handshake e connection termination. Coloque as mensagens envolvidas em uma tabela e, para cada um dos procedimentos, inclua a fonte, o destino, o protocolo e informações das mensagens.}
\paragraph{}Os pacotes do \emph{three-way handshake} são:
\begin{lstlisting}
01:34:41.473518 IP willow.csail.mit.edu.39675 > maple.csail.mit.edu.commplex-link: Flags [S], seq 1258159963, win 14600, options [mss 1460,sackOK,TS val 282136473 ecr 0,nop,wscale 7], length 0
01:34:41.474055 IP maple.csail.mit.edu.commplex-link > willow.csail.mit.edu.39675: Flags [S.], seq 2924083256, ack 1258159964, win 14480, options [mss 1460,sackOK,TS val 282202089 ecr 282136473,nop,wscale 7], length 0
01:34:41.474079 IP willow.csail.mit.edu.39675 > maple.csail.mit.edu.commplex-link: Flags [.], ack 1, win 115, options [nop,nop,TS val 282136474 ecr 282202089], length 0
\end{lstlisting}
\begin{center}
 \begin{tabular}{ | c || c | c | c | c | }
  \hline
  Pacote & Fonte & Destino & Protocolo & Informação \\ \hline \hline
  1 & willow.csail.mit.edu & maple.csail.mit.edu & TCP & SYN \\
  2 & maple.csail.mit.edu & willow.csail.mit.edu & TCP & SYN-ACK \\
  3 & willow.csail.mit.edu & maple.csail.mit.edu & TCP & ACK \\
  \hline
 \end{tabular}
\end{center}

\paragraph{}Os pacotes do término da conexão são:
\begin{lstlisting}
01:34:44.339007 IP maple.csail.mit.edu.commplex-link > willow.csail.mit.edu.39675: Flags [F.], seq 1, ack 1572890, win 905, options [nop,nop,TS val 282204955 ecr 282139320], length 0
01:34:44.339015 IP willow.csail.mit.edu.39675 > maple.csail.mit.edu.commplex-link: Flags [.], ack 2, win 115, options [nop,nop,TS val 282139339 ecr 282204955], length 0
\end{lstlisting}
\begin{center}
 \begin{tabular}{ | c || c | c | c | c | }
  \hline
  Pacote & Fonte & Destino & Protocolo & Informação \\ \hline \hline
  1 & maple.csail.mit.edu & willow.csail.mit.edu & TCP & FIN,  ACK 1572890 \\
  2 & willow.csail.mit.edu & maple.csail.mit.edu & TCP & ACK \\
  \hline
 \end{tabular}
\end{center}



\end{document}
