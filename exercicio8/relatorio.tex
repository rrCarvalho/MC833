\documentclass[a4paper,10pt,oneside,final,titlepage,onecolumn]{article}

\usepackage{ucs}
\usepackage[portuguese]{babel}
\usepackage[utf8x]{inputenc}
\usepackage[T1]{fontenc}
\usepackage{textcomp}
\usepackage{graphicx}
\usepackage{placeins}

\usepackage{listings}
\usepackage{color}

\definecolor{dkgreen}{rgb}{0,0.6,0}
\definecolor{gray}{rgb}{0.5,0.5,0.5}
\definecolor{mauve}{rgb}{0.58,0,0.82}

\lstset{frame=tb,
  language=bash,
  aboveskip=3mm,
  belowskip=3mm,
  showstringspaces=false,
  columns=flexible,
  basicstyle={\scriptsize\ttfamily},
  numbers=none,  
  breaklines=true,
  breakatwhitespace=true
  tabsize=3
}



\title{Exercício 8 de MC833 --- Programação em Redes de Computadores}
\author{Raul Rabelo Carvalho, 105607, turma A}



\begin{document}



\maketitle



\section{}
\paragraph{}Na linguagem Java, os vários passos necessários para obter uma conexão TCP do sistema operacional --- que em C são executados com chamadas de funções em sequência, como \verb|socket()|, \verb|bind|, \verb|connect|, etc. --- são feitos pela implementação da classe java.net.Socket. Assim, no cliente de eco para Android, a conexão com o servidor é estabelecida pela linha:
\\\verb|socket = new Socket(dstAddress, dstPort);|
\\que dá à aplicação um socket conectado ao servidor no endereço \emph{dstAddress} na porta \emph{dstPort}.



\FloatBarrier
\section{}
\paragraph{}A classe java.net.Socket tem diversos construtores diferentes, incluindo\\
\verb|public Socket (String dstName, int dstPort)|,\\
o qual o argumento \verb|String dstName| aceita tanto um \emph{hostname} quanto um endereço IP, contando que o endereço IP seja passado como uma \emph{string} na forma ``x.x.x.x''.



\FloatBarrier
\section{}
\paragraph{}O método \verb|getLocalPort()| da classe java.net.Socket retorna a porta local pela qual o cliente está conectado na forma de uma \emph{string} que pôde ser concatenada com a mensagem indicando uma conexão bem sucedida.



\FloatBarrier
\section{}
\paragraph{}Para implementar a funcionalidade de eco, foi criado um objeto OnClickListener para o botão ``Enviar'', o qual altera uma flag no objeto MyClientTask. No método doInBackground do objeto MyClientTask, foi criado um \emph{loop} que envia ao servidor a \emph{string} do campo TextMessage e recebe a resposta do servidor, imprimindo a resposta.
\begin{lstlisting}
  while (true) {
    if (newline) {
      response = "";
      saida.println(editTextMsg.getText().toString());
      response = entrada.readLine(); 
      publishProgress();
      newline = false;
    }
  }
\end{lstlisting}



\end{document}
